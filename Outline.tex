\documentclass[runningheads,a4paper,oribibl]{llncs}

\usepackage[T1]{fontenc}

\usepackage[utf8]{inputenc}
% \usepackage[latin1]{inputenc}
\usepackage{textcomp}
\usepackage{graphicx}    
\usepackage{url}          

\begin{document}

\pagestyle{headings}

\mainmatter

\title{Exploring the differences in performance between gamers and non-gamers when completing everyday tasks viewed from a third person perspective}


\titlerunning{Third-Person Performance Differences Between Gamers and Nongamers}

\author{Arvid Bräne}

\institute{
	Department of Computing Science \\
	Umeå University, Sweden \\
	\email{arvidbrane@gmail.com} 
}

\maketitle


%\begin{abstract}
%	Here goes the actual text of your abstract.
%\end{abstract}

%\begin{itemize}
%	\item What have I done?
%	\item How did I do it?
%	\item What was the result of the study?
%	\item What to take out from this study
%	\item 2
%	\item 3
%\end{itemize}



\section{Introduction}
The text in this will contain the following:
\begin{itemize}
	\item What have I done?
	\item Why did I do it?
	\item The background to the subject
	\item What is new in this study
\end{itemize}

\begin{description}
   \item[Purpose] To investigate if there is a measurable difference in performance between people whom have played games and people whom have not.
   \item[Motivation] There is often talk about what negative side-effects of playing video games, especially violent ones. My study investigates one of the possible \emph{positive} side-effects.
   \item[Contents] A thural investigation of if there is a side-effect of playing video games viewed in third-person or not.
   \item[Resources] The study has been completed using a custom-made rig consisting of a camera, video goggles, carbon fiber booms, 3D-printed parts, batteries and cables. References to earlier work will also be used.
\end{description}








\section{Method}
In order to see the performance differences between the two groups (gamers and non-gamers) the users will first complete the task just like they do in real life while they are being timed. This time will serve as a baseline for each user. Next up the user will be equipped with a pair of video glasses that are connected to a video camera mounted on a monopod on their back, to simulate the third person perspective that some games offer. 

The users will be recorded and timed while they perform a few (1-3 depending on the time required) tasks (tasks may, or may not, include shopping, cooking food, completing an obstacle course, walking/running, riding a bike, practice a sport, getting dressed etc.). To ensure as high statistical certainty and individual differences all test subjects will be bench-marked against themselves meaning the normal time it took to complete a task (the baseline) will be compared against the time it took with the glasses on.

Both before and after the users will have to fill in a form; the first containing background information (personal, gaming, interest etc.) and the later questions about the what the experience felt like.



This section should contain the following:
\begin{itemize}
	\item Give an overview/introduction over/to how this study was completed
	\begin{itemize}
		\item What kind of tasks
		\item Rig design
		\item Performance benchmarking
	\end{itemize}
	\item References to earlier works
	\item Description about things to take into account
	\item Explaining the form every participant has to fill in

\end{itemize}




\subsection{Test Design}
This section will describe the test-rig and the tasks in detail and;
\begin{itemize}
	\item Number of participants
	\item The design of the tasks (including figures)
	\item The design and purpose of the rig
\end{itemize}





\section{Results}
This section will cover the results from the tests that where done and;
\begin{itemize}
	\item The results from the tests
	\item Diagrams comparing the results
	\item Results of earlier work
	\item Compare the performance between the different groups
\end{itemize}





\section{Discussion}
A general discussion about the study such as;
\begin{itemize}
	\item What part/conclusion in my study could be biast/not reliable
	\item What does my results mean?
	\item What kind of limitations/problems does my solution have?
	\item Earlier work, how do they compare to my work and what does that mean?
\end{itemize}




\subsection{Conclusion}
As a finish, and a complement to the abstract, the conclusion should contain;
\begin{itemize}
	\item What to take out from the study
	\item How this study can be made more in-depth
	\item Future work

\end{itemize}





\nocite{*}
% \bibliographystyle{splncs}
\bibliographystyle{plain-annote}

\bibliography{Bibliography}


\end{document}